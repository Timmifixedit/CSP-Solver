\section{Einführung}
Ein \eac{csp} ist eine Problemstellung, bei der einer Menge von Variablen Werte zugewiesen werden müssen, ohne dass dabei bestimmte Beschränkungen (\textit{Constraints}) 
verletzt werden. Formal besteht ein \ac*{csp} aus einer Menge von Variablen $V$, einer Menge von Wertedomänen $D$ und einer Menge von Constraints $C$. Das \ac*{csp} ist gelöst,
wenn jeder Variable aus $V$ ein Wert aus ihrer zugehörigen Wertedomäne $d \in D$ zugewiesen wurde, sodass alle Constraints aus $C$ erfüllt sind. Ein einfaches Beispiel ist das
Damenproblem, bei dem insgesamt acht Damen so auf einem Schachbrett platziert werden müssen, dass sie sich gegenseitig nicht bedrohen. Hierbei entspricht $V$ den Damen, denen
ein passendes Feld zugewiesen werden muss. Bei jeder Dame umfasst die Wertedomäne $d$ alle möglichen Felder des Schachbretts und die Constraints wurden bereits erwähnt. Neben
dem eben beschriebenen Beispiel finden sich \acp{csp} häufig im Bereich der künstlichen Intelligenz wieder. Beispielsweise werden zum Planen von wissenschaftlichen
Beobachtungsmissionen auf dem Hubble Space Telescope Techniken zum Lösen von \acp{csp} verwendet, um die Betriebszeit und damit Kosten zu reduzieren \cite{HubbleScheduling}.
