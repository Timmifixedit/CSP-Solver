\section{Lösen von \acp{csp}}
\label{sec:Solving}
\acp{csp} kann man in verschiedene Unterklassen unterteilen. Die allgemeinste Form lässt hierbei Variablen mit unendlichen (auch kontinuierlichen) Wertedomänen, soft-Constraints
(also Präferenzen, anstatt Beschränkungen) sowie $n$-äre Constraints zu. Das Lösen eines \ac*{csp} kann im Allgemeinen sehr kompliziert sein, da es sich um ein NP-vollständiges
Problem handelt. In dieser Arbeit werden allerdings einige Einschränkungen gefordert, sodass sich die Implementierung eines Lösers vereinfacht: Zugelassen werden nur Variablem mit
diskreten, endlichen Wertedomänen und harte, binäre Constraints. Letztere sind also Beschränkungen, die von genau zwei Variablen abhängen und in einer gültigen Lösung des \ac*{csp}
erfüllt sein \textit{müssen}. Ein solcher Spezialfall lässt sich dann beispielsweise durch folgenden Algorithmus lösen (gegeben als C++ Pseudoimplementierung):
\lstinputlisting{backtrack.cpp}
Der hier gezeigte Algorithmus ist eine rekursive Backtracking-Suche, die in jedem Rekursionsschritt versucht, einer noch nicht zugewiesenen Variable einen Wert zuzuteilen. Allerdings
macht sich der Algorithmus noch nicht zunutze, dass alle Constraints binär sind. Ganz zu Beginn befindet sich die Abbruchbedingung: Wenn bereits alle Variablen einen gültigen Wert
besitzen und kein Constraint verletzt ist, liefert die Methode \inlcode{cspSolver} \inlcode{true}. Die gültige Zuweisung ist hier im Argument \inlcode{problem} (mit nicht näher
definiertem Datentyp \inlcode{CSP}) gespeichert, das als Referenz übergeben wurde. Ist das Problem noch nicht gelöst, wird die nächste nicht zugewiesene Variable ausgewählt und
alle möglichen Werte der zugehörigen Wertedomäne werden sukzessive ausprobiert. In Zeile 10 wird überprüft, ob die aktuelle Zuweisung erlaubt ist. Im einfachsten Fall erfolgt das
einfach durch Überprüfen sämtlicher Constraints. Falls der Wert kein Constraint verletzt, wird im nächsten Rekursionsschritt die nächste Variable zugewiesen. Sollte es bei einer 
Variable keine gültige Zuweisung geben, ist der Algorithmus in eine Sackgasse gelaufen. Im Suchbaum läuft der Algorithmus dann so lange aufwärts, bis wieder eine gültige Zuweisung
für eine Variable gefunden wird. Gibt es keine solche Variable, dann ist das Problem nicht lösbar, z.B. aufgrund  widersprüchlicher Constraints. 

Wie hier unschwer zu erkennen ist, besitzt diese naive Backtracking-Suche jedoch eine exponentielle Komplexität in der Anzahl der Variablen, was für große \acp*{csp} schnell zu
sehr langen Laufzeiten führen kann. Tatsächlich ist aktuell kein effizienter Algorithmus (also mit polynomieller Laufzeit) bekannt, der ein \ac*{csp}, selbst mit den oben genannten
Einschränkungen, effizient lösen kann, da es sich nach wie vor um ein NP-vollständiges Problem handelt \cite{BestCSPSearch}. Deshalb werden in der Praxis häufig zusätzlich
Heuristiken verwendet, um den Suchraum zu verkleinern und dadurch die Laufzeit zu verringern. Diese Heuristiken können beispielsweise entscheiden, welche Variable als nächstes
ausgewählt werden soll oder welcher Wert als nächstes ausprobiert wird (siehe beispielsweise \cite{OrderingHeuristics}).

\subsection{Lösen durch Constraint Propagation}
\label{sec:ConstrProp}
Ein Problem des zuvor vorgestellten Algorithmus besteht darin, dass die ungültige Zuweisung einer Variable häufig erst viele Rekursionsschritte später bemerkt wird. Dadurch werden
potentiell Äste im Suchbaum verfolgt, die von vornherein schon hätten ausgeschlossen werden könnten. Um dieses Problem zu lösen, bietet sich \textit{Constraint Propagation}
an. Dabei werden aus bereits bestehenden Constraints neue hergeleitet, um die Wertedomänen von Variablen zusätzlich einzuschränken (in \cite{OrderingHeuristics} auch
\textit{Consistency Enforcing} genannt). Dies soll an folgendem Beispiel erläutert werden. Angenommen es gibt zwei Variablen $A$ und $B$, die jeweils Werte zwischen einschließlich
eins und drei annehmen können. Das Constraint sei $A < B$. Daraus lässt sich folgern, dass für $A$ unmöglich der Wert 3 gewählt werden kann, da es in der Domäne von $B$ keinen
größeren Wert gibt. Dieser Wert kann also von vornherein entfernt werden und muss nicht erst ausprobiert werden. Analog kann bei $B$ der Wert 1 aus der Domäne entfernt werden.
In diesem Beispiel stellt sich nach der erläuterten Anpassung der Wertedomänen der Variablen sogenannte \eac{ac} ein. 

\paragraph{\acl*{ac}} Ein \ac*{csp} mit ausschließlich binären Constraints lässt 
sich als Graph darstellen, bei dem die Knoten die Variablen repräsentieren und die Kanten die Constraints. Eine gerichtete Kante zwischen zwei Variablen $A$ und $B$ wird \textit{Arc}
genannt und im Folgenden mit $A \rightarrow B$ bezeichnet. Dies lässt sich interpretieren als: "$A$ fordert von $B$ die Einhaltung eines Constraints". Ein Constraint an sich lässt
sich allgemein als binäres Prädikat angeben und definiert immer beide Richtungen, also $A \rightarrow B$ und $B \rightarrow A$. Bei dem Begriff der \acl*{ac} hingegen spielt die
Richtung eine wichtige Rolle, weshalb zwischen Arcs und Constraints unterschieden werden muss. Formell ist eine Variable $A$ mit Wertedomäne $d_A$ \textit{arc consistent} zu einer
Variablen $B$ mit Wertedomäne $d_B$, wenn $\forall v \in d_A \ \exists w \in d_B$, sodass die Zuweisung $A = v, \ B = w$ keine Constraints zwischen $A$ und $B$ verletzt. Im Folgenden
wird dies bezeichnet durch $A \consTo B$. Hier ist zu beachten, dass im Allgemeinen aus $A \consTo B$ nicht folgt $B \consTo A$. Sind alle alle Arcs zwischen allen Variablenpaaren
eines \ac*{csp} konsistent, dann nennt man das Problem ebenfalls \textit{arc consistent} \cite{ACOverview}. Ein bekannter Algorithmus, der \ac*{ac} in einem \ac*{csp} herstellt,
ist der \ac*{ac}-3 Algorithmus (siehe beispielsweise \cite{ac3}). 

Auch wenn das Herstellen von \ac*{ac} das Problem im Allgemeinen nicht löst, kann dadurch der effektive Suchraum verkleinert werden, wodurch die Suche nach einer Lösung beschleunigt
werden kann. So lässt sich beispielsweise der zuvor angegebene naive Backtracking-Algorithmus folgendermaßen anpassen:
\lstinputlisting{solve.cpp}
Nach der Zuweisung der aktuellen Variable in Zeile acht wird hier durch die Methode \textit{obtainArcConsistency} das Problem auf ein äquivalentes Problem transformiert, wobei
ggf. Werte aus den Domänen mancher Variablen gelöscht werden. Sollte sich die Domäne einer Variablen auf die leere Menge reduzieren, ist das Problem von diesem Rekursionsschritt
aus nicht mehr lösbar, sodass direkt der nächste Wert ausprobiert werden kann. In diesem Fall gibt die Methode \inlcode{false} zurück. Ist das Herstellen der \ac*{ac} erfolgreich,
kann wie zuvor der nächste Rekursionsschritt ausgeführt werden, wobei im Idealfall die Domänen der Variablen reduziert sind. In manchen Fällen lässt sich ein \ac*{csp} sogar
direkt durch das Herstellen von \ac*{ac} lösen.