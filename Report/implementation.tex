\section{Umsetzung}
Ausgehend von den in \cref{sec:Concept} beschriebenen Anforderungen wird im folgenden Teil die Implementierung des generischen \ac*{csp}-Lösers vorgestellt.

\subsection{Zentrale Datenstrukturen}
Um ein \ac*{csp} zu definieren, muss des Nutzer lediglich zwei Dinge tun: Angegeben werden müssen alle vorhandenen Variablen mit zugehöriger Wertedomäne sowie eine Menge von
Constraints, die die Abhängigkeiten zwischen den Variablen darstellen. Dafür werden die Template-Klassen \inlcode{csp::Variable} sowie \inlcode{csp::Constraint} und
\inlcode{csp::Arc} zur Verfügung gestellt.

\subsubsection{csp::Variable}
Die Klasse \inlcode{csp::Variable} ist, wie schon in \cref{sec:CSPDef} beschrieben, sehr einfach gehalten. Sie besitzt einen Member vom Typ \inlcode{std::list}, der die zugehörige
Wertedomäne darstellt. Der Datentyp der Werte in der Domäne lässt sich durch den Template-Typparameter der Klasse angeben. Die Entscheidung ist hier bewusst auf eine
\inlcode{std::list} gefallen, da später durch den \ac*{ac}-3 Algorithmus Werte an beliebigen Stellen in der Domäne gelöscht werden dürfen. Eine \inlcode{std::list} garantiert für
eine solche Operation konstanten Aufwand \cite{stdList}. Da bidirektionales Iterieren nicht benötigt wird, hätte auch eine \inlcode{std::forward\_list} verwendet werden können.
Allerdings stellt diese Datenstruktur keine Methode \inlcode{size()} zur Verfügung. Ein weiterer Kandidat war ein \inlcode{std::unordered_set}. Der Vorteil dabei wäre, dass die
Wertedomäne dann garantiert einzigartige Elemente enthält. Allerdings funktioniert ein \inlcode{std::unordered_set} nur, wenn eine passende Hash-Funktion angegeben wird. Zwar gibt
es mit \inlcode{std::hash} für einige häufig verwendete Datentypen Standardimplementierungen. Wird allerdings ein eigener Datentyp verwendet, so müsste auch explizit eine
Hash-Funktion angegeben werden. Dies wäre auch möglich über einen weiteren Template-Typparameter mit Default-Typ, würde den Nutzer aber trotzdem zur Implementierung einer
Hash-Funktion zwingen, sobald \inlcode{std::hash} keine Spezialisierung für den verwendeten Datentyp anbietet. Damit der Nutzer bei der Angabe der Wertedomäne auch andere
Containertypen verwenden kann, existiert für den Setter \inlcode{setValueDomain} eine Überladung, die beliebige Containertypen annimmt und deren Inhalt in die \inlcode{std::list}
kopiert. Die Container müssen lediglich Iterieren unterstützen, damit das Kopieren mit \inlcode{std::copy} durchgeführt werden kann.

Die Klasse \inlcode{csp::Variable} besitzt keinen eigenen Member, der den zugewiesenen Wert repräsentiert. Eine Variable gilt als zugewiesen, wenn die Wertedomäne genau einen
Wert enthält. Ein Nutzer könnte aber bei Bedarf aus \inlcode{csp::Variable} ableiten und beispielsweise eine Methode \inlcode{getValue()} hinzufügen.

\subsubsection{csp::Constraint und csp::Arc}
\label{sec:ArcConstr}
Constraints werden entweder durch \inlcode{csp::Constraint} oder durch \inlcode{csp::Arc} angegeben. Dabei erbt \inlcode{csp::Arc} aus \inlcode{csp::Constraint}. Der Unterschied
zwischen beiden Klassen wurde bereits in \cref{sec:ConstrProp} erläutert: Constraints geben eine bidirektionale Abhängigkeit zwischen zwei Variablen an, wohin Arcs gerichtet sind.
Beide lassen sich jedoch fast identisch verwenden. Zunächst soll die Klasse \inlcode{csp::Constraint} genauer betrachtet werden. Sie enthält zwei Pointer-Typen, die jeweils auf
eine \inlcode{csp::Variable} zeigen und das Constraint zwischen den Variablen wird durch ein binäres Prädikat in Form einer \inlcode{std::function} repräsentiert. Der genaue Typ
des Pointers wird nicht vorgeschrieben und lässt sich über einen Template-Typparameter angeben. Grundsätzlich sind alle Typen zulässig, die den Dereferenzoperator \inlcode{*}
unterstützen. Möglich sind also beispielsweise einfache "Raw-Pointer" oder auch Smart-Pointer, solange sie den Copy-Konstruktor unterstützen. Alle weiteren Hilfsklassen und
-methoden sowie der Lösealgorithmus selbst verwenden alle denselben Pointer-Typ. Der Pointer selbst muss auf einen Typ zeigen, der aus \inlcode{csp::Variable} ableitet. Verwendet
der Nutzer eine eigene Variablen-Klasse, die aus \inlcode{csp::Variable} erbt und weitere Member besitzt, so ist es weiterhin über den Pointer möglich auf diese Member zuzugreifen,
auch wenn \inlcode{csp::Variable} selbst nicht polymorph ist. Dies ermöglicht mehr Flexibilität beim Schreiben von eigenen Heuristiken zur Variablenauswahl, wie später noch genauer
erläutert wird.

Mithilfe von \acused{sfinae}\ac*{sfinae}-Techniken lässt sich zur Übersetzzeit überprüfen, ob der gegebene Typparameter die oben genannten Anforderungen erfüllt, also
dereferenzierbar ist und auf einen Typ zeigt, der aus \inlcode{csp::Variable} erbt:
\lstinputlisting[linerange={23-36, 41-42, 48-49}]{../src/Arc.h}
Durch \inlcode{static_assert} lassen sich dann aussagekräftige Fehlermeldungen zur Übersetzzeit generieren:
\lstinputlisting[linerange={63-69}]{../src/Arc.h}

Ein \inlcode{csp::Arc} lässt sich identisch zu einem \inlcode{csp::Constraint} konstruieren. Allerdings besitz ein \inlcode{csp::Arc} folgende zusätzliche Member: \inlcode{from()}
liefert einen (allgemeinen) Pointer auf die Variable, von der der Arc ausgeht. Für den Arc $A \rightarrow B$ wäre dies beispielsweise $A$. Analog würde \inlcode{to()} einen Pointer
auf die Variable $B$ liefern. Mithilfe von \inlcode{reverse()} lässt sich der Arc umdrehen. Man würde also den Arc $B \rightarrow A$ erhalten. Beide Arcs wären jedoch nach wie vor
durch dasselbe Constraint repräsentiert. Die Member \inlcode{from()} und \inlcode{to()} stellen dabei unter Beachtung der Richtung des Arcs sicher, dass immer der korrekte Pointer
geliefert wird. Außerdem wird die Auswertung des binären Prädikats durch den Member \inlcode{constraintSatisfied()} vereinheitlicht, der mit zwei Werten aus den Domänen von
\inlcode{from()} und \inlcode{to()} aufgerufen werden sollte und dann sicherstellt, dass die Argumente des Prädikats in der richtigen Reihenfolge angegeben werden.

\subsubsection{csp::Csp}
Die Klasse \inlcode{csp::Csp} repräsentiert ein vollständiges \ac*{csp}. Sie besteht aus einer Liste von Pointern auf Variablen, einer Liste von Arcs und einer Tabelle, die
eingehende Arcs Variablen zuordnet. Analog zu \inlcode{csp::Arc} und \inlcode{csp::Constraint} sind hier der genaue Pointer-Typ und der genaue Variablen-Typ nicht festgelegt.
\inlcode{std::vector} bietet sich für die Liste der Variablen an, da hier keine Elemente hinzugefügt oder gelöscht werden müssen. Der zusammenhängende Speicher ermöglicht außerdem
effizienten Random-Access und effizientes Iterieren. Da der \ac*{ac}-3 Algorithmus auf einer Queue von Arcs arbeitet, bietet sich für die Liste der Arcs eine \inlcode{std::list}
an. Denkbar wäre auch eine \inlcode{std::queue} gewesen. Allerdings unterstützt diese kein Iterieren und kein elegantes Einfügen von Elementen am Ende durch
\inlcode{std::back_inserter} und \inlcode{std::copy}. Für die Zuordnung von eingehenden Arcs zu Variablen wurde eine \inlcode{std::unordered_map} verwendet, die für jede Variable
einen \inlcode{std::vector} aus Arcs verwaltet. Das Finden von Elementen in einer solchen Datenstruktur benötigt im Durchschnitt konstanten Aufwand \cite{uMap}, sodass im \ac*{ac}-3
Algorithmus eingehende Arcs zu einer gegebenen Variable effizient bestimmt werden können. Alle Member von \inlcode{csp::Csp} sind \inlcode{const}, sodass sich die bei der
Konstruktion festgelegte Topologie des Problems nicht mehr ändern lässt. Es ist also nicht möglich, neue Variablen oder Constraints hinzuzufügen oder zu entfernen. Allerdings
lassen sich die Variablen über die Pointer verändern, sodass der Löser später beispielsweise die Wertedomänen anpassen kann. Da die korrekte Zuordnung von eingehenden Arcs zu
Variablen essentiell für die Korrektheit des Lösers ist, darf ein \inlcode{csp::Csp} nur über die \inlcode{friend}-Methode \inlcode{csp::make_csp} konstruiert werden.

\subsection{Definieren eines \ac*{csp}}
Wie schon in \cref{sec:Concept} erörtert, soll der Nutzer lediglich eine Menge von Variablen und Constraints angeben müssen, um ein \ac*{csp} vollständig zu definieren. Hierfür
gibt es die Methode \inlcode{csp::make_csp}, die zwei Container entgegennimmt: Im ersten sind wieder allgemeine Pointer-Typen enthalten, die auf alle Variablen des Problems zeigen.
Im zweiten sind entweder \inlcode{csp::Constraint}s oder \inlcode{csp::Arc}s enthalten. Je nach Problem kann es manchmal einfacher sein, beide Richtungen der Constraints explizit
anzugeben (wie beispielsweise in der Implementierung des Sudoku-Lösers) oder auch nicht. Deshalb werden beide Typen akzeptiert, jedoch keine Mischung aus beiden. Die Containertypen
werden durch Template-Typparameter festgelegt, sodass beliebige Arten von Containern verwendet werden können, die Iteration unterstützen. Werden die Constraints als
\inlcode{csp::Arc}s angegeben, so lässt sich das \inlcode{csp::Csp} ohne großen zusätzlichen Aufwand konstruieren. Lediglich die Variablen mit ihren zugehörigen eingehenden Arcs
müssen in die \inlcode{std::unordered_map} eingetragen werden. Wird stattdessen eine Menge aus \inlcode{csp::Constraint}s angegeben, so müssen diese zuvor noch in ihre äquivalenten
Arcs umgewandelt werden. Dies geschieht durch die Memberfuntion \inlcode{getArcs()} von \inlcode{csp::Constraint}.

Da beide Überladungen von \inlcode{csp::make_constraint} eine identische Signatur besitzen (bis auf die Namen der Template-Typparameter), muss \ac*{sfinae} eingesetzt werden, damit
die Wahl der richtigen Methode zur Übersetzzeit eindeutig ist. Dazu besitzen beiden Methoden einen dritten Template-Parameter, der lediglich dazu dient, die unpassende Methode
auszuschließen:
\lstinputlisting[linerange={94-97}]{../src/Csp.h}
Das Konstrukt nutzt \inlcode{std::enable_if_t}, um einen Template-Parameter vom Typ \inlcode{int} mit Standardwert 0 zu deklarieren, sodass bei korrekter Benutzung der Methode
der dritte Parameter nicht explizit angegeben werden muss. Sollte der im Typ \inlcode{ArcContainer} enthaltene Typ nicht vom Typ \inlcode{csp::Arc} sein, so definiert
\inlcode{std::enable_if_t} keinen Typ und der Ausdruck des Standardwerts ist nicht gültig. Dadurch wird diese Überladung von \inlcode{csp::make_csp} durch \ac*{sfinae}
ausgeschlossen. Bei der Überladung der Methode, die \inlcode{csp::Constraint}s annimmt, wird ein analoger Test durchgeführt, sodass maximal eine der beiden Überladungen ausgewählt
werden kann. Die dafür verwendeten Type-Traits sind ähnlich aufgebaut wie jene, die in \cref{sec:ArcConstr} vorgestellt wurden.

\subsection{Lösealgorithmus}
Mithilfe der zuvor dargestellten Datenstrukturen lässt sich der eigentliche Lösealgorithmus recht einfach implementieren. Der während der Backtracking-Suche verwendete \ac*{ac}-3
Algorithmus ist quasi eine Eins-zu-eins-Implementierung des Pseudocodes aus \cite{ac3}. Der Unterschied ist, dass in der hier vorgestellten Implementierung nur binäre Constraints
erlaubt sind. Unäre Constraints lassen sich allerdings trivial auf die Wertedomäne einer Variable übertragen, sodass diese Aufgabe an den Nutzer delegiert wurde.
Die Backtracking-Suche selbst ist in der Methode \inlcode{csp::recursiveSolve} implementiert und ist ebenfalls sehr ähnlich zu der in \cref{sec:ConstrProp} vorgestellten
Pseudoimplementierung. Auch sie arbeitet direkt auf einer Referenz des Problems und verändert die Domänen der Variablen über die enthaltenen Pointer. Ob das gegebene
(Teil-)Problem erfolgreich gelöst werden konnte, wird auch hier durch einen Boolean signalisiert. Ein paar Unterschiede gibt es dennoch:
\lstinputlisting[linerange={29-35}]{../src/csp_solver.h}
Zusätzlich zum \ac*{csp} muss ein Strategie-Funktor übergeben werden, der die nächste, noch nicht zugewiesene Variable aus dem \inlcode{csp::Csp} auswählt. Es werden beliebige
Funktoren akzeptiert, die eine korrekte Signatur besitzen. Somit erhält der Nutzer die Möglichkeit, beliebige, problemspezifische Heuristiken zur Auswahl der nächsten Variablen
anzugeben. Werden eigene Variablentypen verwendet, die aus \inlcode{csp::Variable} ableiten, kann eine benutzerdefinierte Heuristik aufgrund des statischen Polymorphismus auch auf
dessen zusätzliche Felder und Methoden zugreifen, sodass mehr Informationen bei der Auswahl der nächsten Variable zur Verfügung stehen. Es wird gefordert, dass die Heuristik
immer eine Variable liefert. Im Falle, dass es keine nicht zugewiesene Variable mehr gibt, muss eine beliebige gewählt werden. Dadurch vereinfacht sich der Goal-Test zu der
einfachen \inlcode{if}-Abfrage in Zeile fünf. Danach werden für die aktuelle Variable sukzessive Werte aus ihrer Wertedomänen ausprobiert:
\lstinputlisting[linerange={38-40, 43-56}, firstnumber=8]{../src/csp_solver.h}
Da die Zuweisung eines Wertes durch \inlcode{assign} die Wertedomäne einer Variable verändert, muss diese zunächst gesichert werden, damit Iterieren weiterhin möglich ist.
Außerdem werden vor dem Aufruf des \ac*{ac}-3 Algorithmus alle Wertedomänen aller Variablen gesichert, da diese durch das Herstellen von \ac*{ac} oder den folgenden
Rekursionsschritt verändert werden können. Dies geschieht durch \inlcode{util::makeCspCheckpoint}. Der ursprüngliche Zustand lässt sich danach durch
\inlcode{util::restoreCspFromCheckpoint} wiederherstellen.

Anstatt für \inlcode{csp::recursiveSolve} einen in-out-Parameter zu nutzen, also das Problem als nicht konstante Referenz zu übergeben, hätte auch ein reiner in-Parameter verwendet
werden können, der nicht verändert werden darf. Eine einfache \inlcode{const}-Referenz reicht dabei nicht aus, da sich die Wertedomänen der Variablen dann trotzdem über die
Pointer verändern lassen. Um zu gewährleisten, dass das ursprüngliche Problem wirklich unverändert bleibt, muss vor dem Anpassen der Variablen eine Deep-Copy des \inlcode{csp::Csp}
erzeugt werden. Dadurch wird dann gleichzeitig das Sichern der Wertedomänen durch \inlcode{util::makeCspCheckpoint} überflüssig. Allerdings ist das Erstellen einer Deep-Copy nicht
trivial, da nicht nur der Speicher hinter den Variablen-Pointern kopiert werden muss, sondern auch die zum Problem zugehörigen \inlcode{csp::Arc}s neu erstellt werden müssen. Eine
Version des Lösers mit in-Parameter wurde ebenfalls implementiert \footnote{Der Code findet sich auf \url{https://github.com/Timmifixedit/CSP-Solver/tree/DeepCopy} auf dem
experimentellen \textit{DeepCopy}-Branch}. Hier liefert die Backtracking-Suche bei Erfolg eine Kopie des Problems zurück, bei der alle Variablen korrekt belegt sind. Auch der
\ac*{ac}-3 Algorithmus arbeitet nicht mehr direkt auf dem Problem, sondern erstellt ebenfalls eine Kopie, wodurch das Sichern der Wertedomänen nicht mehr benötigt wird.
\inlcode{csp::Csp} besitzt eine Methode \inlcode{clone()}, die die Deep-Copy erstellt. Experimente mit dem Sudokulöser zeigen jedoch, dass diese Version des Lösealgorithmus, selbst
mit der gcc-Optimierung \inlcode{-O3}, um bis zu Faktor drei langsamer ist als die Version mit in-out-Parameter und ohne Deep-Copy \footnote{Die Laufzeiten wurden beim Lösen des
sehr schwierigen Sudokus gemessen, das im Ordner \textit{res} beigefügt ist}. Im Folgenden wird deshalb weiterhin die ursprüngliche Version des Lösers vorgestellt.

Zur einfacheren Verwendung wird mit \inlcode{csp::solve} ein Wrapper für die rekursive Backtracking-Suche bereitgestellt:
\lstinputlisting[linerange={69-82}]{../src/csp_solver.h}
Hier wird außerdem mithilfe des \inlcode{static_assert} und \inlcode{std::is_invocable_r_v} zur Übersetzzeit überprüft, ob die angegebene Heuristik eine gültige Signatur besitzt.
Standardmäßig wird die bereits implementierte Heuristik \inlcode{strategies::Mrv} verwendet. Diese wählt die Variable mit den wenigsten möglichen Werten aus (\inlcode{Mrv} steht
für \textit{Minimum Remainig Values}). Eine weitere Möglichkeit ist die \inlcode{strategies::First} Heuristik, die einfach die nächstbeste nicht zugewiesene Variable auswählt.
Darüber hinaus lassen sich aber auch beliebige eigene Heuristiken verwenden, solange die Signatur zulässig ist. Außerdem prüft der Wrapper anfangs, ob das Problem leer ist und ruft
einmal den \ac*{ac}-3 Algorithmus auf, da die rekursive Suche davon ausgeht, dass das Problem bereits \textit{arc consistent} ist. Sollte schon dieser erste Schritt fehlschlagen,
ist das Problem nicht lösbar und die Backtracking-Suche obsolet.
