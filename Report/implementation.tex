\section{Umsetzung}
Ausgehend von den in \cref{sec:Concept} beschriebenen Anforderungen wird im folgenden Abschnitt die Implementierung des generischen \ac*{csp}-Lösers vorgestellt.

\subsection{Zentrale Datenstrukturen}
Um ein \ac*{csp} zu definieren, muss des Nutzer lediglich zwei Dinge tun: Alle vorhandenen Variablen angeben (mit zugehöriger Wertedomäne), sowie eine Menge von Constraints, die
die Abhängigkeiten zwischen den Variablen darstellen. Dafür werden die Template-Klassen \inlcode{csp::Variable}, sowie \inlcode{csp::Constraint} oder \inlcode{csp::Arc} zur
Verfügung gestellt.

\subsubsection{csp::Variable}
Die Klasse \inlcode{csp::Variable} ist, wie schon in \cref{sec:CSPDef} beschrieben, sehr einfach gehalten. Sie besitzt einen Member vom Typ \inlcode{std::list}, der die zugehörige
Wertedomäne darstellt. Der Datentyp der Werte in der Domäne lässt sich durch den Template-Typparameter der Klasse angeben. Es wurde sich bewusst für eine \inlcode{std::list}
entschieden, da später durch den \ac*{ac}-3 Algorithmus Werte an beliebigen Stellen in der Domäne gelöscht werden dürfen. Eine \inlcode{std::list} garantiert für eine solche
Operation konstanten Aufwand \cite{stdList}. Da bidirektionales Iterieren nicht benötigt wird, hätte auch eine \inlcode{std::forward\_list} verwendet werden. Allerdings stellt
diese Datenstruktur keine Methode \inlcode{size()} zur Verfügung. Es gibt keinen eigenen Member, der den Wert einer zugewiesenen Variable repräsentiert. Eine Variable zählt als
zugewiesen, wenn die Wertedomäne genau einen Wert enthält. Ein Nutzer könnte aber bei Bedarf aus \inlcode{csp::Variable} ableiten und beispielsweise eine Methode \inlcode{getValue()}
hinzufügen.

\subsubsection{csp::Constraint und csp::Arc}
\label{sec:ArcConstr}
Constraints werden entweder durch \inlcode{csp::Constraint} oder durch \inlcode{csp::Arc} angegeben. Dabei erbt \inlcode{csp::Arc} aus \inlcode{csp::Constraint}. Der Unterschied
zwischen beiden Klassen ist der bereits in \cref{sec:ConstrProp} erläuterte Unterschied zwischen Constraints, die eine bidirektionale Abhängigkeit zwischen zwei Variablen angeben,
und Arcs, die gerichtet sind. Beide lassen sich jedoch fast identisch verwenden. Zunächst soll die Klasse \inlcode{csp::Constraint} genauer betrachtet werden: Sie enthält zwei
Pointer-Typen, die jeweils auf eine \inlcode{csp::Variable} zeigen und das Constraint zwischen den Variablen wird durch ein binäres Prädikat in Form einer \inlcode{std::function}
repräsentiert. Der genaue Typ des Pointers wird nicht vorgeschrieben und lässt sich über einen Template-Typparameter angeben. Grundsätzlich sind alle Typen zulässig, die den
Dereferenzoperator \inlcode{*} unterstützen. Möglich sind also beispielsweise einfache "Raw-Pointer" oder auch Smart-Pointer, solange sie den Copy-Constructor unterstützen. Alle
weiteren Hilfsklassen und -Methoden, sowie der Lösealgorithmus selbst verwenden diesen angegeben Pointer-Typ, um das Kopieren von Variablen zu vermeiden. Der Pointer selbst muss
auf einen Typ zeigen, der aus \inlcode{csp::Variable} ableitet. Verwendet der Nutzer eine eigene Variablen-Klasse, die aus \inlcode{csp::Variable} erbt und weitere Member besitzt, 
so ist es weiterhin möglich auf diese Member über den Pointer zuzugreifen, auch wenn \inlcode{csp::Variable} selbst nicht polymorph ist. Dies ermöglicht mehr Flexibilität beim
schreiben von eigenen Heuristiken zur Variablenauswahl, wie später noch genauer erläutert werden soll.

Mithilfe von \acused{sfinae}\ac*{sfinae}-Techniken lässt sich zur Übersetzzeit überprüfen, ob der gegebene Typparameter die oben genannten Anforderungen erfüllt (also
dereferenzierbar und zeigt auf einen Typen, der aus \inlcode{csp::Variable} erbt):
\lstinputlisting[linerange={23-50}]{../src/Arc.h}
Durch \inlcode{static_assert} lassen sich dann aussagekräftige Fehlermeldungen zur Übersetzzeit generieren:
\lstinputlisting[linerange={63-69}]{../src/Arc.h}

Ein \inlcode{csp::Arc} lässt sich identisch zu einem \inlcode{csp::Constraint} konstruieren. Allerdings besitz ein \inlcode{csp::Arc} folgende zusätzliche Member: \inlcode{from()}
liefert einen (allgemeinen) Pointer auf die Variable, von der der Arc ausgeht. Für den Arc $A \rightarrow B$ wäre dies beispielsweise $A$. Analog würde \inlcode{to()} einen
Pointer auf $B$ liefern. Mithilfe von \inlcode{reverse()} lässt sich der Arc umdrehen. Man würde also den Arc $B \rightarrow A$ erhalten. Beide Arcs würden jedoch nach wie vor
durch dasselbe Constraint repräsentiert werden. Die Member \inlcode{from()} und \inlcode{to()} stellen dabei unter Beachtung der Richtung des Arcs sicher, dass immer der korrekte
Pointer geliefert wird. Ebenfalls wird die Auswertung des binären Prädikats durch den Member \inlcode{constraintSatisfied()} vereinheitlicht, der mit zwei Werten aus den Domänen
von \inlcode{from()} und \inlcode{to()} aufgerufen werden sollte und dann sicher stellt, dass die Argumente des Prädikats in der richtigen Reihenfolge angegeben werden.

\subsubsection{csp::Csp}
Die Klasse \inlcode{csp::Csp} repräsentiert ein vollständiges \ac*{csp}. Sie besteht aus einer Liste aus Pointern auf Variablen, einer Liste aus Arcs und einer Tabelle, die
eingehende Arcs zu Variablen zuordnet. Analog zu \inlcode{csp::Arc} und \inlcode{csp::Constraint} ist hier der genaue Pointer-Typ und der genaue Variablen-Typ nicht festgelegt.
\inlcode{std::vector} bietet sich für die Liste der Variablen an, da hier keine Elemente hinzugefügt oder gelöscht werden müssen. Der zusammenhängende Speicher ermöglicht außerdem
effizienten Random-Access und effizientes Iterieren. Da der \ac*{ac}-3 Algorithmus auf einer Queue von Arcs arbeitet, bietet sich für die Liste der Arcs eine \inlcode{std::list}
an. Denkbar wäre auch eine \inlcode{std::queue} gewesen. Allerdings unterstützt diese kein Iterieren und kein elegantes Einfügen von Elementen am Ende durch
\inlcode{std::back_inserter} und \inlcode{std::copy}. Zuordnung zwischen Variablen und eingehenden Arcs wurde eine \inlcode{std::unordered_map} verwendet, die für jede Variable
einen \inlcode{std::vector} aus Arcs verwaltet. Das finden von Elementen in einer solchen Datenstruktur benötigt im Durchschnitt konstanten Aufwand \cite{uMap}, sodass eingehende
Arcs zu einer gegebene Variable im \ac*{ac}-3 Algorithmus effizient bestimmt werden können.

\subsection{Definieren eines \ac*{csp}}
Wie schon in \cref{sec:Concept} erörtert, soll der Nutzer lediglich eine Menge von Variablen und Constraints angeben müssen, um ein \ac*{csp} vollständig zu definieren. Hierfür
gibt es die Methode \inlcode{csp::make_csp}, die zwei Container entgegen nimmt: Im ersten sind wieder allgemeine Pointer-Typen enthalten, die auf alle Variablen des Problems zeigen.
Im zweiten sind entweder \inlcode{csp::Constraint}s oder \inlcode{csp::Arc}s enthalten. Je nach Problem kann es manchmal einfacher sein, beide Richtungen der Constraints explizit
anzugeben (wie beispielsweise in der Implementierung des Sudoku-Lösers), oder auch nicht. Deshalb werden beide Typen akzeptiert, jedoch keine Mischung aus beiden. Die Containertypen
werden durch Template-Typparameter festgelegt, sodass beliebige Arten von Containern verwendet werden, die Iteration unterstützen. Werden die Constraints als \inlcode{csp::Arc}s
angegeben, so lässt sich das \inlcode{csp::Csp} ohne großen zusätzlichen Aufwand konstruieren. Lediglich die Variablen mit ihren zugehörigen eingehenden Arcs müssen in die
\inlcode{std::unordered_map} eingetragen werden. Wird stattdessen eine Menge aus \inlcode{csp::Constraint}s angegeben, so müssen diese zuvor noch in ihre äquivalenten Arcs
umgewandelt werden. Dies geschieht durch die Memberfuntion \inlcode{getArcs()} von \inlcode{csp::Constraint}.

Da beide Überladungen von \inlcode{csp::make_constraint} eine identische Signatur besitzen (bis auf die Namen der Template-Typparameter), muss \ac*{sfinae} eingesetzt werden, damit
die Wahl der richtigen Methode zur Übersetzzeit eindeutig ist. Dazu besitzen beiden Methoden einen dritten Template-Parameter, der lediglich dazu dient, die unpassende Methode
auszuschließen:
\lstinputlisting[linerange={75-78}]{../src/Csp.h}
Das Konstrukt nutzt \inlcode{std::enable_if_t}, um einen Template-Parameter vom Typ \inlcode{int} mit Standardwert 0 zu deklarieren, sodass bei korrekter Benutzung der Methode
der dritte Parameter nicht explizit angegeben werden muss. Sollte der im Typ \inlcode{ArcContainer} enthaltene Typ nicht vom Typ \inlcode{csp::Arc} sein, so definiert
\inlcode{std::enable_if_t} keinen Typ und der Ausdruck des Standardwerts ist nicht gültig. Dadurch wird diese Überladung von \inlcode{csp::make_csp} durch \ac*{sfinae}
ausgeschlossen. Bei der Überladung der Methode, die \inlcode{csp::Constraint}s annimmt, wird ein analoger Test durchgeführt, sodass maximal eine der beiden Überladungen ausgewählt
werden kann. Die dafür verwendeten Type-Traits sind ähnlich aufgebaut wie jene, die in \cref{sec:ArcConstr} vorgestellt wurden.