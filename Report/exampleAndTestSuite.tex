\section{Testsuite und Beispielanwendung}
Nachdem ein Überblick über den Aufbau des Lösealgorithmus gegeben wurde, sollen nun kurz die Implementierung eines Sudokulösers sowie die Testsuite vorgestellt werden.

\subsection{Lösen von Sudokus}
Die bekannten Sudoku-Rätsel lassen sich recht einfach als \ac*{csp} beschreiben. Dabei stellen die einzelnen Felder die Variablen dar, die mit Zahlen von eins bis neun ausgefüllt
werden müssen. Die Constraints äußern sich darin, dass Felder in derselben Zeile, Spalte oder im selben $3 \times 3$ Block unterschiedliche Werte haben müssen. Für den Löser wurde
ein eigener Variablen-Typ erstellt, der im Konstruktor automatisch die Wertedomäne korrekt setzt. Außerdem lässt sich der Wert der Variable bequem mit \inlcode{getValue()} abfragen:
\lstinputlisting[linerange={14-29}]{../src/main.cpp}
Um die \inlcode{SudokuNode}s zu verwalten, wurde außerdem die Hilfsklasse \inlcode{Sudoku} implementiert. Deren Konstruktor liest die gegebenen Werte aus einem \inlcode{std::istream}
ein und instanziiert automatisch alle Variablen mit den korrekten Werten. Der Wert 0 gibt hierbei an, dass ein Sudoku-Feld noch leer ist. Außerdem lässt sich das Sudoku mit
\inlcode{print()} anzeigen und mit \inlcode{solve()} lösen:
\lstinputlisting[linerange={94-106}]{../src/main.cpp}
Zunächst müssen alle Abhängigkeiten zwischen den Feldern angegeben werden. Diese werden im \inlcode{std::vector arcs} gespeichert. Das Verwenden eines \inlcode{std::array} ist
nicht möglich, da ein \inlcode{csp::Arc} nicht default-konstruierbar ist. Danach wird das zu lösende Problem durch \inlcode{csp::make_csp} definiert und anschließende durch
\inlcode{csp::solve} gelöst. Hier wird explizit die Heuristik \inlcode{csp::strategies::First} verwendet, die bei Experimenten mit verschiedenen Sudokus bessere Performance gezeigt
hat als die \inlcode{csp::strategies::Mrv} Strategie. Nach dem Aufruf von \inlcode{solve()} kann, falls lösbar, die Lösung des Sudokus erneut mit \inlcode{print()} angezeigt werden.

\subsection{Unit-Tests mit Google Test}
Während der Entwicklung wurden insgesamt 21 Unit-Tests mit dem Google-Test-Framework geschrieben, um die korrekte Funktionalität der einzelnen Komponenten zu gewährleisten.
Der Aufbau eines solchen Unit-Test soll an einem Beispiel kurz dargestellt werden:
\lstinputlisting[linerange={25-32}]{../Tests/util.cpp}
Ein einzelner Test wird mit dem Makro \inlcode{TEST} definiert, wobei ein Name für die übergeordnete Testgruppe und ein spezifischer Test-Name angegeben werden muss. Innerhalb des
Test-Blocks wird dann der Code für den Test platziert. Der hier gezeigte Test überprüft, ob die Methode \inlcode{util::removeInconsistent}, die im \ac*{ac}-3 Algorithmus zum Einsatz
kommt, den korrekten Wert zurückliefert. Dies lässt sich mit den von Google Test zur Verfügung gestellten Makros wie \inlcode{EXPECT_TRUE} überprüfen. Es gibt darüber hinaus
weitere Makros, die auch komplexeres Verhalten überprüfen können, beispielsweise auch, ob eine bestimmte Exception ausgelöst wurde.
